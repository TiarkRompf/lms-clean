\section{Normal Frontend}

After discussing the simple frontend in Section~\ref{sec:s_front}, we want to show
what a normal frontend looks like (in the object Adaptor at core/stub.scala).

object Adapter extends Frontend
   typeMap
   funTable

   emitCommon // code gen ???

   class MyGraphBuilder extends GraphBuilder

\subsection{Base: Introducing Rep[T]}

In the Base trait, the code establish the iconic Rep[T] of LMS.
Previously in Frontend class, we have seen one way to wrap around
core.backend.Exp so that we can construct LMS Graph via unary operators,
binary operators, and et al. What is to be further provided in Base trait
is the ability to use Rep[T]. Similarly Rep[T] is built on top of
core.backend.Exp. The core.backend.Exp do not have types. The types are
added via user code and type inferencing, and then tracked in a data structure
called typeMap (such as Adaptor.typeMap).

\begin{listing}[scala]
trait Base extends EmbeddedControls with OverloadHack with ClosureCompare {
    type Rep[+T] = Exp[T]  // type name aliasing

    abstract class Exp[+T] // track LMS IR for non-variables
    abstract class Var[T]  // track LMS IR for variables

    // The Wrap class and method that build Rep[T] typed expression with type tracking
    case class Wrap[+A:Manifest](x: lms.core.Backend.Exp) extends Exp[A] {
        Adapter.typeMap(x) = manifest[A]
    }
    def Wrap[A;Manifest](x: lms.core.Backend.Exp): Exp[A] = {
        if (manifest[A] == manifest[Unit]) Const(()).asInstanceOf[Exp[A]]
        else new Wrap[A](x)
    }

    // The WrapV class for Var[T]
    case class WrapV[A:Manifest](x: lms.core.Backend.Exp) extends Var[A] {
        Adapter.typeMap(x) = manifest[A]
    }
}
\end{listing}

\subsection{Base: Better Handling of Functions}

In the simple frontend, we see that the handling of recursive functions is
a bit awkward. How do we make it better with a better frontend?
In Base trait, we express the type of in-graph function better:

\begin{listing}[scala]
def fun[A:Manifest, B:Manifest](f: Rep[A]=>Rep[B]): Rep[A=>B] =
    Wrap[A=>B](__fun(f, 1, xn => Unwrap(f(Wrap[A](xn(0))))))
\end{listing}

Unfortunately, we have to implement multiple @fun@s for different function
arities, which we will elide here. The @__fun@ function reifies the argument
into a function block.

\begin{listing}[scala]
def __fun[T:Manifest](f: AnyRef, arity: Int, gf: List[Backend.Exp] => Backend.Exp): Backend.Exp = {
    // use canonicalize to get the unique representation of any Scala function
    val can = canonicalize(f)
    Adapter.funTable.find(_._2 == can) match {
        case Some((funSym, _)) =>
            funSym // Easy case: found the function in funTable
        case _ =>
            // Step 1. set up "lambdaforward" node with 2 new fresh Syms
            val fn = Backend.Sym(Adapter.g.fresh)
            val fn1 = Backend.Sym(Adapter.g.fresh)
            Adapter.g.reflect(fn, "lambdaforward", fn1)()

            // Step 2. register (fn, can) in funTable, so that recursive calls
            //    will find fn as the function Sym. Reify the block.
            // Note: it might seem strange why/how recursive calls re-enter this __fun() function.
            //    The reason is that in user code, recursive functions have to be written as
            //    lazy val f = fun{...} or def f = fun{...}, in which case the recursive calls
            //    will re-enter the `fun` call.
            Adapter.funTable = (fn, can)::Adapter.funTable
            val block = Adapter.g.reify(arity, gf)

            // Step 3. build the "lambda" node with fn1 as the function name
            //    fix the funTable such that it pairs (fn1, can) for non-recursive uses.
            val res = Adapter.g.reflect(fn1,"lambda",block)(hardSummary(fn))
            Adapter.funTable = Adapter.funTable.map {
            case (fn2, can2) => if (can == can2) (fn1, can) else (fn2, can2)
            }
            res
    }
}
\end{listing}

Although the @__fun@ function provided a nice solution to recursive functions, it does
add complexity to code generations and LMS graph transformations, which we will cover later.
The Base trait also provides @topFun@, which is a variant of @fun@ that is supposed to be lifted
to top level (for C code generation). There are still many limitations to topFun related to closure
conversion, recursion, and so on.

The Base trait also provide macro support for native @if@, @while@, and @var@, by extending
the @EmbeddedControls@ and providing the following methods:
\begin{listing}[scala]
def __ifThenElse[T:Manifest](c: Rep[Boolean], a: => Rep[T], b: => Rep[T]): Rep[T]
def __whileDo(c: => Rep[Boolean], b: => Rep[Unit]): Rep[Unit]
def var_new[T:Manifest](x: Rep[T]): Var[T]
def __assign[T:Manifest](lhs: Var[T], rhs: T): Unit
\end{listing}
With the macro (\at virtualize), native @if@, @while@, and @var@ are syntactically transformed
to these functions, which are then converted to @IF@, @WHILE@, and WrapV[T].

The Base trait also support misc ops, boolean ops, timing ops, comment ops, unchecked ops, and
et al.

\subsection{Other Rep[T] Handling}

Since a Type T object can have many methods, we need to support those methods for Type Rep[T]
objects. The way is to implement
\begin{enumerate}
\item implicit conversion from T, Rep[T], and Var[T] to a class (TOpsCls) wrapping Rep[T].
\item all the desired methods in that class (TOpsCls).
\end{enumerate}

An example of ordering is given below. All these traits have to extend Base or be implemented
in Base trait directly, to be able to use Rep[T]. This pattern of code is used for many types
that are supported in LMS IR as Rep[T], including primitives (such as Int, Float, and Double) and
some data structures (such as List, Tuple, Array, and Map)

\begin{listing}[scala]
trait OrderingOps extends Base with OverloadHack {
  implicit def orderingToOrderingOps[T:Ordering:Manifest](n: T) = new OrderingOpsCls(unit(n))
  implicit def repOrderingToOrderingOps[T:Ordering:Manifest](n: Rep[T]) = new OrderingOpsCls(n)
  implicit def varOrderingToOrderingOps[T:Ordering:Manifest](n: Var[T]) = new OrderingOpsCls(readVar(n))

  class OrderingOpsCls[T:Ordering:Manifest](lhs: Rep[T]){
    def <  (rhs: Rep[T])(implicit pos: SourceContext) =
      Wrap[Boolean](Adapter.g.reflect("<", Unwrap(lhs), Unwrap(rhs)))
    def <= (rhs: Rep[T])(implicit pos: SourceContext) =
      Wrap[Boolean](Adapter.g.reflect("<=", Unwrap(lhs), Unwrap(rhs)))
  }
\end{listing}


